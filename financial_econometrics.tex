\documentclass[
    russian
    ]{vegalectures}

\usepackage{algpseudocode}
\usepackage{appendix}

\title{Финансовая эконометрика}
\author{Станкевич Иван Павлович}
\date{Осенний семестр 2023--2024}

\begin{document}
    \maketitle

    \begin{abstract}
        Курс финансовой эконометрики знакомит с основными классами моделей, использующихся в финансах. Предполагается, что слушатели знакомы с эконометрикой, но ещё незнакомы с моделями временных рядов. Курс начинается с основных концепций анализа временных рядов -- сюда входят базовые одномерные модели (ARIMA и ETS), коинтеграция, базовые многомерные модели. В дальнейшем рассматриваются специфичные для финансовой сферы вопросы, такие как CAPM-модель, моделирование волатильности, нелинейные модели временных рядов.
    \end{abstract}

    \tableofcontents

    \section{Основные понятия теории временных рядов}
    
    \section{Классический подход к теории временных рядов}
    \subsection{Автокорреляционная функция и частная автокореляционная функция}
        Важный инструмент исследования временного ряда.
        \begin{definition}
            Пусть $Y_t$ -- стационарный в широком смысле процесс. Тогда ее автокореляционная функция определяется как 
            \begin{equation*}
                \rho_k = \frac{\cov \left(Y_t, Y_{t-k}\right)}{\var Y_t}.
            \end{equation*}
        \end{definition}

        \paragraph{Уравнения Юла-Уокера}
            Рассмотрим стационарное решение процесса $AR(1)$:
            \begin{equation*}
                y_{t} = \alpha y_{t-1} +\epsilon_t,\quad |\alpha| < 1.
            \end{equation*}
            \begin{equation*}
                \mathbb{E}\left[y_t y_{t-1}\right] = \mathbb{E}\left[(\alpha y_{t-1} +\epsilon_t) y_{t-1}\right],
            \end{equation*}
            откуда
            \begin{align*}
                & \gamma_0 = \alpha \gamma_1 + \sigma_\epsilon^2, \\
                & \gamma_1 = \alpha\gamma_0,
            \end{align*}
            где
            \begin{equation*}
                \gamma_k = \cov\left(y_t, y_{t-k}\right).
            \end{equation*}
            Итого, 
            \begin{align*}
                & \gamma_0 = \frac{\sigma_\epsilon^2}{1-\alpha^2}, \\
                & \gamma_1 = \alpha\frac{\sigma_\epsilon^2}{1-\alpha^2}, \\
                & ... \\
                & \gamma_k = \alpha^{k}\frac{\sigma_\epsilon^2}{1-\alpha^2}, \\
                & ...
            \end{align*}
            Получившиеся уравнения называются \emph{уравнениями Юла-Уокера}.
            Аналогично для стационарного решения $MA(1)$, $y_t = \epsilon_t + \beta \epsilon_{t-1}$:
            \begin{align*}
                & \gamma_0 = , \\
                & \gamma_1 = \beta\sigma_\epsilon^2, \\
                & \gamma_2 = 0, \\
                & ... \\
                & \gamma_k = 0, \\
                & ...
            \end{align*}

        \begin{proposition}
            Для стационарного решения $MA(q)$ $ACF(k) \equiv 0$ при $k>q$.
        \end{proposition}

        \begin{definition}
            Пусть $Y_t$ -- стационарный в широком смысле процесс. Тогда её частная автокореляционная функция определяется как коэффициент $\phi_{k,k}$ из регрессии
            \begin{equation*}
                y_t = \phi_{k,1} y_{t-1} +\dots+ \phi_{k,k} y_{t-k} + u_t.
            \end{equation*}
        \end{definition}

        \begin{example}[$PACF$ для процесса $AR(1)$]
            \begin{align*}
                & \phi_{k, 1} = \alpha, \\
                & \phi_{k, 2} = 0, \\
                & ... \\
                & \phi_{k, k} = 0.
            \end{align*}
            Итого, $PACF(k) = \alpha \delta_1^k$, где $\delta_i^j$ -- символ Кронекера.
        \end{example}

    \subsection{Методология Бокса и Дженкинса (1970)}
        Авторами предлагаются следующие шаги\footnote{Первые 2 шага называются идентификацией, третий -- оцениванием.}:
        \begin{enumerate}
            \item Ряд остационариваем разностями ($y_t \mapsto \Delta^d y_t$, оператор первой разности действует следующим образом: $\Delta y_t = y_t - y_{t-1}$);
            \item Смотрим на $ACF$ и $PACF$ \footnote{TODO: расписать случаи}:
                    \begin{itemize}
                        \item $AR(p)$
                        \item $MA(q)$
                        \item $ARMA(p,q)$.
                    \end{itemize}
            \item Оцениваем $ARIMA(p,d,q)$.
        \end{enumerate}

        Так уже никто не делает, но это важный этап развития эконометрической мысли. Выяснилось, что не всегда можно остационаривать ряд разностями (например, ряды с экспоненциальными трендами); стационарность определяется большим количеством тестов.
        Более того, для подбора параметров $ARIMA$ можно использовать информационные критерии AIC/BIC: чем больше значение статистики, тем хуже модель.

    \subsection{Сезонность временных рядов}
        Сезонная модификация модели $ARIMA$ -- $SARIMA(p, d, q)\times (P, D, Q)_S$ имеет следующий вид:
        \begin{multline}
            \Delta^d \Delta_S^D y_{t} =
                    \alpha_1 \Delta^d \Delta_S^D y_{t-1} + \dots + \alpha_p \Delta^d \Delta_S^D y_{t-p} + \\ +
                    \alpha_1^S \Delta^d \Delta_S^D y_{t-1} + \dots + \alpha_P^S \Delta^d \Delta_S^Dy_{t-PS} + \\ +
                    \beta_1 \Delta^d \Delta_S^D \epsilon_{t-1} + \dots + \beta_q \Delta^d \Delta_S^D \epsilon_{t-q} + \\ +
                    \beta_1^S \Delta^d \Delta_S^D \epsilon_{t-1} + \dots + \beta_Q^S \Delta^d \Delta_S^D \underbrace{\epsilon_{t-QS}}_{\text{<<$Q$ лет назад>>}},
        \end{multline} 
        где $\Delta_S = (1-L^S)$ -- сезонный оператор первой разности. 

    \subsection{Тесты на единичные корни}
        Это семейство тестов проверяет принадлежность корней характеристического уравнения единичной окружности.

        \paragraph{Тест Дики-Фуллера (Dickey-Fuller test)}
            Рассмотрим АР1
            \begin{align*}
                & y_t = \alpha y_{t-1} + \epsilon_t\qquad | -y_{t-1} \\
                & \Delta y_t = \hat\beta \Delta y_{t-1}, \\
                & H_0: \beta = 0\text{ vs }H_1: \beta < 0,\\
                & \text{Статистика } t = \frac{\hat\beta}{\hat{SE}(\hat\beta)}.
            \end{align*}

        \paragraph{Расширенный тест Дики-Фуллера (Augmented DF test)}
            Все ровно то же самое, но добавили предыдущие разности:
            \begin{align*}
                & \Delta y_t = ... + (\alpha-1) y_{t-1} + \beta_1 \Delta y_{t-1} \dots + \beta_k \Delta y_{t-k} + \epsilon_t, \\
                & H_0: \beta = 0\text{ vs }H_1: \beta < 0,\\
                & \text{Статистика } t = \frac{\hat\beta}{\hat{SE}(\hat\beta)}.
            \end{align*}

        \paragraph{Тест КПСС (KPSS test, 1993)}
        Пример теста на стационарность вокруг тренда (trend stationarity).
            \begin{align*}
                & y_t = \underbrace{C_t}_{C_t = C_{t-1}+u_t} + \underbrace{\gamma t}_{\text{Линейный тренд}} + \epsilon_t, \quad \epsilon_t \independent u_t \\
                & H_0: \var u_t = 0 \text{ vs } H_1: \var u_t > 0,\\
            \end{align*}

        \paragraph{Тест Льюнга-Бокса (Ljung-Box test)}
            Вычислим по выборке $\hat \rho_1, \dots, \hat \rho_k$. $H_0: \hat \rho_1 = \dots = \hat \rho_k = 0$. Статистика:
            \begin{equation*}
                Q = T(T+2)\sum_{i=1}^k \frac{\hat \rho_i^2}{n-i} \sim \chi^2_{k}.
            \end{equation*}
            

            
    \section{Exponential Smoothing (ETS)}
    \subsection{Simple ES}
        \begin{align*}
            & \hat y_{T+1\vert T} = y_T\\
            &\ || \\
            & \alpha y_T + \alpha(1-\alpha)y_{T-1} + \dots.
        \end{align*}
    \subsection{Component form}
        \noindent Forecast equation: $\hat y_{T+h\vert T} = l_T$.

        \noindent Level equation: $l_t = \alpha y_t + (1-\alpha)l_{t-1}$.

    \subsection{Тренд Holt (1957)}
        \noindent Forecast equation: $\hat y_{T+h\vert T} = l_T + hb_T$.

        \noindent Level equation: $l_t = \alpha y_t + (1-\alpha)(l_{t-1}+b_{t-1})$.

        \noindent Trend equation: $b_t = \beta(l_t - l_{t-1}) + (1-\beta) b_{t-1}$.

    \subsection{Тренд с сезонностью Holt-Winters}
        \noindent Forecast equation: $\hat y_{T+h\vert T} = l_T + hb_T + S_{t+h - s(k-\cdot)}$.

        \noindent Level equation: $l_t = \alpha (y_t - S_{t-s}) + (1-\alpha)(l_{t-1}+b_{t-1})$.

        \noindent Trend equation: $b_t = \beta(l_t - l_{t-1}) + (1-\beta) b_{t-1}$.

        \noindent Seasonality equation: $S_t = \gamma(y_t - l_{t-1} - b_{t-1}) + (1-\gamma) S_{t-s}$.

    \subsection{Damped trend}
        \noindent Forecast equation: $\hat y_{T+h\vert T} = l_T + {\color{red}(\phi + \phi^2 + \dots + \phi^h)}b_T + S_{t+h - s(k-\cdot)}$, $\phi \in (0, 1)$.

        \noindent Level equation: $l_t = \alpha (y_t - S_{t-s}) + (1-\alpha)(l_{t-1}+b_{t-1})$.

        \noindent Trend equation: $b_t = \beta(l_t - l_{t-1}) + (1-\beta) b_{t-1}$.

        \noindent Seasonality equation: $S_t = \gamma(y_t - l_{t-1} - b_{t-1}) + (1-\gamma) S_{t-s}$.
    
    \subsection{Prophet}
        \begin{equation*}
            y_t = \underbrace{g_t}_{\text{Trend}} + \underbrace{S_t}_{\underbrace{\text{Seasonality}}_{\text{сезон, недельная, внутридневная}}} + \underbrace{h_t}_{\text{Holidays}} + e_t
        \end{equation*}

    \subsection{Кросс-валидация моделей временных рядов}
        \begin{algorithmic}
            \State $y_t, t\in\{0, \dots, N\}$
            \State $n \gets n_0$.
            \While {$n+h \leq N$}
                \State Строим $\hat y_{n+h|n} = \hat y_n (y_{n-n_0}, \dots y_{n})$
                \State $n \gets n+1$
            \EndWhile
        \end{algorithmic}

        Другой алгоритм:
        \begin{algorithmic}
            \State $y_t, t\in\{0, \dots, N\}$
            \State $n \gets 0$.
            \While {$n \leq H$}
                \State Строим $\hat y_{t|t-n}$
                \State $n \gets n+1$
            \EndWhile
        \end{algorithmic}

        \textbf{Метрики}: MAE\footnote{Не чувствителен к разовым выбросам}, RMSE\footnote{Сильно чувствителен к единичным выбросам}, MAPE\footnote{Очищен от масштаба}, MASE\footnote{Относительное сравнение модели со случайным блужданием}.

    \subsection{Model selection}
        \subsubsection{Тест Diebold-Mariano (1995)}
            Сравнивает прогнозную силу моделей. Пусть $g(\cdot)$ -- некоторая функция потерь.
            \begin{enumerate}
                \item $e_t^A = y_t - \hat y_t^A$, $e_t^B = y_t - \hat y_t^B$;
                \item $d_t =g(e^A_t) - g(e^B_t)$
            \end{enumerate}
            \noindent $H_0: \mathbb{E}\left[d_t\right] = 0$ vs $H_A: \mathbb{E}\left[d_t\right] \neq 0$. Статистика выглядит следующим образом:
            \begin{align*}
                & \bar d = \frac{1}{k}\sum_t d_t,\quad \bar f = \sum_j \gamma_d(j), \\
                & S = \frac{\bar d}{\sqrt{\bar f/k}} \overset{H_0}{\sim} N(0, 1).
            \end{align*}

    \section{Многомерные модели}
    Если мы рассмотрим 2 независимых ряда, стационарных вокруг разных линейных 
    трендов (но не стационарных), то выборочная корреляция будет значимо отлична от нуля. 
    Поэтому в этой лекции рассматриваем стационарные модели рядов.
    В финансовой эконометрике доходности активов $\frac{y_t - y_{t-1}}{y_{t-1}}$ обычно почти что стационарны.

    Начнем с обобщений модели ARIMA.

    \subsection{ARIMAX and Intervention analysis}
        \begin{definition}[$\mathrm{ARIMAX}(p,d,q)$]
            \begin{multline}
                \Delta^d y_{t} =
                        \alpha_1 \Delta^d y_{t-1} + \dots + \alpha_p \Delta^d y_{t-p} + 
                        \beta_1 \Delta^d \epsilon_{t-1} + \dots \\ \dots + \beta_q \Delta^d \epsilon_{t-q} + 
                        \gamma_1 X_{1t} + \dots + \gamma_k X_{kt},
            \end{multline} 
            где $X_{it}$ -- значения стационарного ряда в какой-то момент времени (можем при помощи $i$ сдвигать время).
        \end{definition}

        Рассмотрим $\mathrm{AR}(1)$ модель с простым структурным срывом:
        \begin{align}
            & y_t = \alpha_0 + \alpha_1 t_{t-1} + c z_t + \epsilon_t, \\
            & z_t = \begin{cases}
                1,& t\geq t^*,\\
                0,& t<t^*.
            \end{cases}
        \end{align}
        Мгновенный эффект $+c$. Каков будет долгосрочный эффект? 
        \begin{align}
            & \text{До: }&\mathbb{E} \left[y_t\right] &= \alpha_0 + \alpha_1 \mathbb{E} \left[y_{t-1}\right], \\
            & \text{После: }&\mathbb{E} \left[y_t\right] &= (\alpha_0+c) + \alpha_1 \mathbb{E} \left[y_{t-1}\right].
        \end{align}
        Долгосрочный эффект: $+\frac{c}{1-\alpha_1}$ (если брать различие по сравнению с моментом структурного срыва).
        \begin{definition}
            Impulse response function -- функция следующего вида:
            \begin{equation*}
                \mathrm{IRF} = \frac{\partial y_t}{\partial z_{t^*}}.
            \end{equation*} 
            Показывает как меняется временной ряд в долгосрочной перспективе (явно выражаем $y_t = y_{t^*+(t-t^*)}$).
        \end{definition}
        Рассмотрим 
        \begin{equation}
            y_t = \alpha(L) y_t + \gamma(L)X_t + \epsilon_t,
        \end{equation}
        где $\alpha$, $\gamma$ -- лаговые полиномы.
        Выряжая $y$, получаем 
        \begin{equation}
             y_t = (I - \alpha(L))^{-1} \gamma(L)X_t + (I - \alpha(L))^{-1} \epsilon_t.
        \end{equation}
        Отсюда можно найти $\mathrm{IRF}$ по любому из $X$. Когда мы так делаем, мы предполагаем значимое влияние $X$ на $y$, 
        при этом $X$ -- экзогенный ряд. Но есть проблема: а что если причинность другая? Симс: "это все фигня, го ботать векторную авторегрессию".

    \subsection{Vector autoregression}
        \begin{definition}[$\mathrm{VAR}(p)$]
            Пусть $y_t = [y^1_t, \dots, y^k_t]^T$. Модель векторной авторегрессии\footnote{Интерпретируем это как <<всё в прошлом влияет на всё в настоящем>>} $p$-ого порядка выглядит следующим образом:
            \begin{equation}
                y_t = A_0 + A_1y_{t-1}+ \dots + A_py_{t-p} + \epsilon_t.
            \end{equation}
            Оценка происходит построчно при помощи МНК.
        \end{definition}
        Симсон предложил: мы в VAR слишком много фигни оцениваем, нам важно получить состоятельные оценки, а не эффективные.
        \begin{example}[Двумерный $\mathrm{VAR}(1)$]
            \begin{align}
                & x_t = a_0 + a_1 x_{t-1} + a_2 z_{t-1} + \epsilon^1_t, \\
                & z_t = b_0 + b_1 x_{t-1} + b_2 z_{t-1} + \epsilon^2_t.
            \end{align}
        \end{example}
        Рассмотрим $\mathrm{VAR}(1)$:
        \begin{align*}
            & y_t = A_0 + A_1y_{t-1} + \epsilon_t\\
            & (I - A_1L) y_t = A_0 + \epsilon_t\\
            &  y_t = (I - A_1L)^{-1}A_0 + (I - A_1L)^{-1}\epsilon_t\\
            &  y_t = \widetilde A_0 + \underbrace{A_1^n y_{t-n}}_{\to 0} + \sum_{i=1}^n A_1^i \epsilon_{t-i} \to \mathrm{VMA}(\infty)\\
        \end{align*}

        
    \section{Байесовская векторная авторегрессия}
    Замечание: BVAR -- некоторое обобщение регуляризации. Байесовский VAR -- это про очень большие модели с недостатком данных для точной фреквентивистской оценки.
    \subsection{Minnesota Prior}
        \begin{definition}[Minnesota Prior]
            Положим $\mathrm{VAR}(p): \quad Y_t = X_t \Phi + \epsilon_t$, $\phi := \operatorname{vec} \Phi$, $\epsilon \sim \mathcal{N}(0; \Sigma)$.
            \begin{equation*}
                \mathrm{Prior}\cdot \phi \sim \mathcal{N}(\underline{\phi}; \underline{\Xi}).
            \end{equation*}
            Тогда 
            \begin{equation*}
                \mathrm{Posterior}\cdot \phi^{\text{post}} \sim \mathcal{N}(\overline{\phi}; \overline{\Xi}).
            \end{equation*}
            Рассмотрим $\underline \phi = \mathbb{E}^{\text{prior}}\left[\phi\right] = \operatorname{vec}\underline{\Phi}$\footnote{восстановим $\underline{\Phi}$ как обращение оператора $\mathrm{vec}$}.
            Говорим, что модель имеет априорное распределение Минессоты, если 
            \begin{equation*}
                (A_l)_{ij} = \delta\mathbbm{1}(l=1, i=j), \qquad \delta \in [-1, 1].
            \end{equation*}
        \end{definition}

        \begin{remark}
            Смысл этого априорного распределения: пусть все компоненты рядов $\mathrm{AR}(1)$.
        \end{remark}

        \begin{example}
            $X\sim U[0, \Theta]$. Prior: $\Theta \sim \mathrm{Pareto}(\alpha, \beta)$, \begin{equation*}
                f(\theta) = \begin{cases}
                    \frac{\alpha \beta^\alpha}{\theta^{\alpha+1}}, & \theta > \beta > 0, \\
                    0, \text{ else}.
                \end{cases}
            \end{equation*}
            Posterior: \begin{equation*}
                f^{\text{post}}(\theta) =\mathcal{L} f(\theta) = f(\theta) \prod_{i=1}^{n}\frac{1}{\theta} = \frac{\alpha \beta^\alpha}{\theta^{\alpha+n+1}} \sim \mathrm{Pareto}(\alpha^*, \beta^*),
            \end{equation*}
            где $\alpha^* = \alpha + n$, $\beta^* = \max(\beta,\max X)$.
        \end{example}
        Далее все $\lambda$ -- гиперпараметры модели.
        \begin{align*}
            &\underline\Xi = \left[\begin{matrix}
                 \Xi_\text{const} & 0 & 0 &  & 0 \\
                 0 & \Xi_1 & 0 & \dots & 0 \\
                 0 & 0 & \Xi_2 &  & 0 \\
                  & \dots &  & \dots &  \\
                 0 & 0 & 0 &  & \Xi_{p} \\
            \end{matrix}\right], \\
            &(\Xi_{i, l})_{jj} = \begin{cases}
                \frac{\lambda_{\text{tight}}}{l^{\lambda_{\text{lag}}}}, & i=j, \\
                \frac{(\lambda_{\text{tight}} \lambda_{\text{kron}} \sigma_i)^2}{(l^{\lambda_{\text{lag}}} \sigma_j)^2}, & i\neq j.
            \end{cases}\qquad {l = 1, \dots, p}, \\
            & \Xi_{i,\text{const}} = (\lambda_{\text{tight}} \lambda_{\text{const}} \sigma_i)^2.
        \end{align*}

    \subsection[Декомпозиция дисперсии (FEVD)]{Forecasting Error Variance Decomposition}
        \begin{align*}
            & \hat Y_{T+1} = \hat A_0 + \hat A_1 Y_T + \dots + \hat A_p Y_{T-p+1}, \\
            &  Y_{T+1} =  A_0 +  A_1 Y_T + \dots +  A_p Y_{T-p+1} + \epsilon_{T+1}.
        \end{align*}
        Если оценка хорошая, то она несмещенная и 
        \begin{equation*}
            \hat e_{T+1} = Y_{T+1} - \mathbb{E}\left[\hat Y_{T+1}\right] = \epsilon_{T+1}.
        \end{equation*}
        Аналогично, 
        \begin{equation*}
            \hat e_{T+2} = A_1 \epsilon_{T+1} + \epsilon_{T+2}.
        \end{equation*}
        Итого, 
        \begin{equation*}
            \hat e_{T+h; j} = a_{11}\epsilon_{1;T+1} + a_{1n}\epsilon_{1 ; T+ n} + \dots + a_{k1}\epsilon_{k ; T+1 } + \dots,
        \end{equation*}
        где $j$ -- номер $y$. То есть можем разложить итоговую ошибку как комбинацию ошибок прошлого. Поскольку можем рассматривать ортогонализированный шум, то получаем выражение для дисперсии ошибки.




    \newpage\clearpage
    \begin{appendices}
    \section{Bootstrap sampling}
        Пусть $X_1, \dots, X_n$ -- выборка. Генерим семплы размера $n$ из этой выборки (равномерно по имеющимся значениям). Делаем $B\sim 10^3$ т.н. bootstrap samples.
        По каждой подвыборке оцениваем регрессию и получаем $B$ разных оценок $\beta_i$ в $y = \beta_0 + \beta_1 X$.

        Для временных рядов: как попало семплить не можем, т.к. данные упорядоченны во времени.
        Делим ряд на набор блоков и применяем к ним bootstrap sampling. Если блоки маленькие, то зависимости во времени примерно сохраняются. 
        Такой подход называется \emph{block bootstrap sampling}.

    \section{Информационные критерии}
        Информационные критерии: чем меньше число, тем лучше модель.
        \begin{definition}[Информационный критерий Акаике]
            Пусть $k$ -- число оцениваемых параметров модели, $\mathcal{L}^* = \mathcal{L}(\hat\theta)$ -- правдоподобие полученной оценки. Тогда 
            \begin{equation*}
                \mathrm{AIC} = 2k - 2\ln \mathcal{L}^*.
            \end{equation*}
        \end{definition}

        \begin{definition}[Информационный критерий Шварца-Байеса]
            Пусть $k$ -- число оцениваемых параметров модели, $n$ -- число наблюдений, $\mathcal{L}^* = \mathcal{L}(\hat\theta)$ -- правдоподобие полученной оценки. Тогда 
            \begin{equation*}
                \mathrm{BIC} = k\ln n - 2\ln \mathcal{L}^*.
            \end{equation*}
        \end{definition}

        
\end{appendices}


        


\end{document}